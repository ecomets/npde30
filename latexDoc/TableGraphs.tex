\documentclass[11pt,a4paper]{report}
\usepackage{fontenc}
 
\usepackage[utf8]{inputenc}
\usepackage[frenchb]{babel}
\NoAutoSpaceBeforeFDP
\usepackage{sectsty}
\usepackage{anysize} 
\marginsize{2.25cm}{2.25cm}{2.25cm}{2.25cm}
%\usepackage{fullpage}
\usepackage{graphicx} 
\usepackage{array}
\usepackage[all]{xy}
\usepackage{amsmath,bm}
\usepackage{amssymb}
\usepackage{amsfonts}
\usepackage{mathrsfs}
\usepackage{import}
\usepackage{array}
\usepackage{nth}
\usepackage{listings}
\usepackage{enumitem}
\usepackage{floatrow}
\usepackage{longtable}
\usepackage{titlesec}
\setlist{nolistsep}
\usepackage{cases} 
\usepackage{subfigure}
\usepackage{epstopdf} 
\usepackage[table]{xcolor}
\usepackage{booktabs,multirow,bigstrut}
\usepackage{nameref}
\usepackage{nomencl}
\usepackage{fancyhdr}
\usepackage[export]{adjustbox}
\usepackage{tikz}
\usetikzlibrary{calc}
\usepackage{gensymb}
\usepackage{tabularx}
\usepackage{enumitem}
\usepackage{pdflscape}
\usepackage[most]{tcolorbox}
\usepackage{multirow}
\usepackage{rotating}
\usepackage{amssymb}
 
\usepackage{floatrow}
\usepackage{pdfpages}
\usepackage{ threeparttable}
\usepackage{titlesec}

\floatsetup[table]{capposition=top}
\newcommand*{\Rttensor}[1]{\overline{\overline{#1}}}
\newcommand*\underdot[1]{
\underaccent{\dot}{#1}}
\newcolumntype{R}[1]{>{\Raggedleft\arraybackslash}p{#1}}
\renewcommand*{\thesubfigure}{}
\usepackage{etoolbox}
\usepackage{lmodern}
\setcounter{tocdepth}{1}     
\setcounter{secnumdepth}{0}  

%\usepackage[labelformat=empty]{caption}
\usepackage{tikz-cd}
\usepackage{tikz}
\usepackage{pgfplots}
\usepackage[normalem]{ulem}
\usepackage{microtype}
\bibliographystyle{apalike}
\usepackage[square,sort,comma]{natbib}

\pgfplotsset{every axis/.append style={                
                    label style={font=\Large},
                    tick label style={font=\Large}  
                    }}
\usepackage{titlesec}

\titleformat{\chapter}[display]
{\normalfont\Large\bfseries}{\chaptertitlename\ \thechapter}{0pt}{\Large}
\renewcommand{\thechapter}{\arabic{chapter}}
\addto\captionsfrench{\renewcommand{\chaptername}{Axe}}
\titleclass{\chapter}{straight}
\titlespacing*{\chapter}{0pt}{18pt}{12pt} 
 
\usepackage[labelsep=period]{caption}
\captionsetup[table]{name=TABLE}
\renewcommand{\thetable}{\Roman{table}}
\lstset{basicstyle=\ttfamily\normalsize,breaklines=TRUE}

\renewcommand{\familydefault}{\sfdefault}
\renewcommand{\arraystretch}{1.5}

\captionsetup{font=small}
\newcolumntype{C}{>{\centering\arraybackslash}X}

\usepackage{soul}

\definecolor{NumCol}{rgb}{0.686,0.059,0.569}
\definecolor{commentstyleCol}{rgb}{0.678,0.584,0.686}
\definecolor{keywordstyleCol}{rgb}{0.737,0.353,0.396}
\definecolor{stringstyleCol}{rgb}{0.192,0.494,0.8}

%\lstset{basicstyle=\ttfamily\footnotesize,breaklines=TRUE}

\lstset{backgroundcolor=\color{gray!30}}
\setlist{nolistsep}
\lstset{frame=tb,
language=R,  
numbersep=5pt,
showspaces=FALSE,               % show spaces adding particular underscores
showstringspaces=FALSE,         % underline spaces within strings
showtabs=FALSE,
% keywordstyle=\color{keywordstyleCol},      % keyword style
  commentstyle=\color{commentstyleCol},   % comment style
  stringstyle=\color{stringstyleCol},      % string literal style
  literate=%
   *{0}{{{\color{NumCol}0}}}1
    {1}{{{\color{NumCol}1}}}1
    {2}{{{\color{NumCol}2}}}1
    {3}{{{\color{NumCol}3}}}1
    {4}{{{\color{NumCol}4}}}1
    {5}{{{\color{NumCol}5}}}1
    {6}{{{\color{NumCol}6}}}1
    {7}{{{\color{NumCol}7}}}1
    {8}{{{\color{NumCol}8}}}1
    {9}{{{\color{NumCol}9}}}1
}

\usepackage{fancyhdr}
 
\fancypagestyle{plain}{%
   \fancyhf{}
   \fancyfoot[C]{\iffloatpage{}{\thepage}}
   \renewcommand{\headrulewidth}{0pt}}
\pagestyle{plain}

%%%%%%%%%%%%%%%%%%%%%%%%%%%%%%%%%%%%%%%%%%%%%%%%%%%%%%%%%%%%%%%%%%%%%%%%%%%%%%%%%%%%%%%%%%%%%%%%%%%%%%%%%%%%%%%%%%%%%%%%%%%%%%%%%%%%%%%%%%%%%%%%%%%%%%%%%%%%%%%%%%%%%%%%%%%%%%%%%%%%%%%%%%%
\begin{document}


\subsection{ Types of graphs}



Table \ref{tab::types_of_plot} shows which plot types are available (some depend on whether for instance covariates or data below the limit of quantification are present in the dataset) for a {\sf NpdeObject} object. Given an object {\sf x} resulting from a call to {\sf npde} or {\sf autonpde}, default plots can be produced using the following command:

\begin{lstlisting}
plot(x)
\end{lstlisting}
Different plots are also available using the option {\sf plot.type}, as in:
\begin{lstlisting}
plot(x,plot.type="data")
\end{lstlisting} 

 
\begin{center}
\begin{table}[h!]
\begin{tabularx}{\textwidth}[t]{|C p{9cm} c|}
\arrayrulecolor{black}\hline
\centering{\textbf{\textcolor{black}{Plot types}} }& \textbf{\textcolor{black}{Description types}}\\
\hline
\ttfamily{data} & Plots the observed data in the dataset \\
\ttfamily{x.scatter} & Scatterplot of the npde~versus the predictor X (optionally can plot pd~or npd~instead) \\
\ttfamily{pred.scatter} & Scatterplot of the npde~versus the population predicted values \\
%cov.scatter & Scatterplot of the npde~versus covariates \\
\ttfamily{vpc} & Plots a Visual Predictive Check \\
\ttfamily{loq} & Plots the probability for an observation to be BQL, versus the predictor X \\
\ttfamily{ecdf} & Empirical distribution function of the npde (optionally pd~or npd) \\
\ttfamily{hist} & Histogram of the npde (optionally pd~or npd) \\
\ttfamily{qqplot} & QQ-plot of the npde versus its theoretical distribution (optionally pd~or npd) 
\\
%& \\
\st{cov.x.scatter}& Scatterplot of the npde~versus the predictor X, split by covariate & \textcolor{blue}{Quid ?}\\
\st{cov.pred.scatter} & Scatterplot of the npde~versus the population predicted values, split by ovariate \\
\st{cov.ecdf} & Empirical distribution function of the npde (optionally pd~or npd), split by covariate \\
\st{cov.hist} & Histogram of the npde (optionally pd~or npd), split by covariate \\
\st{cov.qqplot} & QQ-plot of the npde versus its theoretical distribution (optionally pd~or npd), split by covariate \\
\arrayrulecolor{black}\hline
\end{tabularx}
\caption{Types of plots available.}
\label{tab::types_of_plot}
\end{table}
\end{center}

\noindent The final five plots can also be accessed with the base plot and the option: \begin{lstlisting} 
covsplit=TRUE 
\end{lstlisting}
For instance, the following instruction: 
\begin{lstlisting}
plot(x,plot.type="cov.x.scatter")
\end{lstlisting}
is equivalent to:
\begin{lstlisting}
plot(x,plot.type="x.scatter",covsplit=TRUE)
\end{lstlisting}

 
\subsection{Options for graphs}
\begin{sloppypar}
Default layout for graphs in the \textbf{npde library} can be modified through the use of many options. An additional document, \textbf{demo\_npde2.0.pdf}, is included in the \textbf{inst directory} of the package, presenting additional examples of graphs and how to change the options. Table \ref{tab::default_graph_param} following table shows the options that can be set, either by specifying them on the fly in a call to plot applied to a NpdeObject object, or by storing them in the \textbf{prefs} component of the object. Note that not all of the graphical parameters in \texttt{par()} can be used, but it is possible for instance to use the {\sf xaxt="n"} option below to suppress plotting of the X-axis, and to then add back the axis with the R~function {\sf axis()} to tailor the tickmarks or change colours as wanted. It is also possible of course to extract npde, fitted values or original data to produce any of these plots by hand if the flexibility provided in the library isn't sufficient. Please refer to the document \verb+demo_npde2.0.pdf+ for examples of graphs using these options.
\end{sloppypar}

\begin{table}[H] 
\begin{center}
%\begin{tabular}{r p{10cm} p{3cm}}
\begin{tabularx}{\textwidth}[t]{|C p{7cm} C c|}
\arrayrulecolor{black}\hline
& \centering {\textbf{\textcolor{black}{General graphical options}}} & \\
\centering{\textbf{\textcolor{black}{Plot types}} }& \centering{\textbf{\textcolor{black}{Description }}} & \textbf{\textcolor{black}{Default value}} \\
\hline
%\multicolumn{3}{l}{{\itshape \bfseries General graphical options}} \\

\st{\ttfamily{ new}} & Whether a new plot should be produced  & TRUE \\
\st{\ttfamily ask} & Whether users should be prompted before each new plot (if TRUE) & FALSE \\
{\ttfamily verbose} & Output is produced for some plots (most notably when binning is used, this prints out the boundaries of the binning intervals) if TRUE & FALSE \\
{\ttfamily xaxt} & A character which specifies the x axis type. Specifying "n" suppresses plotting of the axis & empty & \textcolor{blue}{à garder} \\
{\ttfamily yaxt} & A character which specifies the y axis type. Specifying "n" suppresses plotting of the axis & empty & \textcolor{blue}{à garder} \\
\st{\ttfamily frame.plot} & If TRUE, a box is drawn around the current plot & TRUE \\

{\ttfamily main} & Title & empty & \textcolor{blue}{\ttfamily garder}\\
\textcolor{red}{\ttfamily main.title} & Main title & & \textcolor{blue}{main}\\
\textcolor{red}{\ttfamily sub.title } & Title for covariate & & \textcolor{blue}{sub}\\
\textcolor{red}{\ttfamily size.main.title } & Size of the main title  & & \textcolor{blue}{size.main}\\
\textcolor{red}{\ttfamily size.sub.title  } & Size of the title for covariate &  & \textcolor{blue}{size.sub}\\


{\ttfamily xlab} & Label for the X-axis & \textcolor{blue}{depends on plot} \\
{\ttfamily ylab} & Label for the Y-axis & \textcolor{blue}{depends on plot} \\
\textcolor{red}{\ttfamily size.xlab } & Size of the label for the X-axis &\\
\textcolor{red}{\ttfamily size.ylab } &  Size of the label for the Y-axis &\\

{\ttfamily xlog} & Scale for the X-axis (TRUE: logarithmic scale) & FALSE \\
{\ttfamily ylog} & Scale for the Y-axis (TRUE: logarithmic scale) & FALSE \\

\st{\ttfamily cex} & A numerical value giving the amount by which plotting text and symbols should be magnified relative to the default & 1 \\
\st{\ttfamily cex.axis} & Magnification to be used for axis annotation relative to the current setting of 'cex' & 1 \\
\st{\ttfamily cex.lab} & Magnification to be used for x and y labels relative to the current setting of 'cex' & 1 \\
\st{\ttfamily cex.main} & Magnification to be used for main titles relative to the current setting of 'cex' & 1 \\

\hline
\end{tabularx} 
\end{center}
\end{table} 

%%%%%%%%%%%%%%%%%%%%%%%%%%%%%%%%%%%%%%%%%%%%%%%%%%%%%%%%%%%%%%%%%%%%%%%%%%%%%%%%%%%%%%
%%%%%%%%%%%%%%%%%%%%%%%%%%%%%%%%%%%%%%%%%%%%%%%%%%%%%%%%%%%%%%%%%%%%%%%%%%%%%%%%%%%%%%

\begin{table}[h!] 
\begin{center}
\begin{tabularx}{\textwidth}[t]{|C p{7cm} C c|}
\arrayrulecolor{black}\hline
& \centering {\textbf{\textcolor{black}{General graphical options (continued)}}} & \\
\centering{\textbf{\textcolor{black}{Plot types}} }& \centering{\textbf{\textcolor{black}{Description }}} & \textbf{\textcolor{black}{Default value}} \\
\hline
\st{\ttfamily mfrow} & Page layout (NA: layout set by the plot function or before) & NA \\
\textcolor{red}{\ttfamily grid.arrange  } & Size of the title for covariate &\\
{\ttfamily xlim} & Range for the X-axis (NA: ranges set by the plot function) & NA \\
{\ttfamily ylim} & Range for the Y-axis (NA: ranges set by the plot function) & NA \\
\st{\ttfamily type} & Type of plot ("b": both, "p": points, "l": lines). Defaults to b for data and p for other plots & b/p & \textcolor{blue}{à garder} \\
\textcolor{red}{\ttfamily  size.text.x } & Size for the numbers on the X-axis & \\
\textcolor{red}{\ttfamily  size.text.y } & Size for the numbers on the Y-axis & \\
    
\textcolor{red}{\ttfamily  x.breaks } & Number of ticks for the X-axis & & \textcolor{blue}{breaks.x}\\
\textcolor{red}{\ttfamily  y.breaks } & Number of ticks for the Y-axis & & \textcolor{blue}{breaks.y}\\
    
\textcolor{red}{\ttfamily  plot.loq } & Plot for the abline at the value of the loq & & \textcolor{blue}{doublon (dans le code aussi)}\\
& & \\
\hline
%\\
\end{tabularx} 
\end{center}
\end{table}

%%%%%%%%%%%%%%%%%%%%%%%%%%%%%%%%%%%%%%%%%%%%%%%%%%%%%%%%%%%%%%%%%%%%%%%%%%%%%%%%%%%%%%
%%%%%%%%%%%%%%%%%%%%%%%%%%%%%%%%%%%%%%%%%%%%%%%%%%%%%%%%%%%%%%%%%%%%%%%%%%%%%%%%%%%%%%


%%%%%%%%%%%%%%%%%%%%%%%%%%%%%%%%%%%%%%%%%%%%%%%%%%%%%%%%%%%%%%%%%%%%%%%%%%%%%%%%%%%%%%
%%%%%%%%%%%%%%%%%%%%%%%%%%%%%%%%%%%%%%%%%%%%%%%%%%%%%%%%%%%%%%%%%%%%%%%%%%%%%%%%%%%%%%

\begin{table}[H] 
\begin{center}
%\begin{tabular}{r p{10cm} p{3cm}}
\begin{tabularx}{\textwidth}[t]{|C p{9cm} C|}
\arrayrulecolor{black}\hline
& \centering {\textbf{\textcolor{black}{Options controlling the type of plots}}} & \\
\centering{\textbf{\textcolor{black}{Parameter}} }&Number of  label for the X-axis \centering{\textbf{\textcolor{black}{Description }}} & \textbf{\textcolor{black}{Default value}} \\
\hline
{\ttfamily plot.type} & Type of plot (see documentation for list) & default \\
{\ttfamily ilist} & List of subjects to include in the individual plots & 1:N \\
\st{\ttfamily smooth} & Whether a smooth should be added to certain plots & FALSE \\
\st{\ttfamily line.smooth} & Type of smoothing (l=line, s=spline) & s \\
{\ttfamily which.cov} & Which covariates to use for the plot  & all \\
{\ttfamily ncat} & Number of categories in which to split continuous covariates for graphs & 3 \\
& Defaults to 3, splitting in $<$Q$_1$, Q$_1$-Q$_3$, $>$Q$_3$ & \\
\st{\ttfamily which.resplot} & Type of residual plot ("res.vs.x": scatterplot \& c("res.vs.x","res.vs.pred", 
\&  versus X, "res.vs.pred": scatterplot versus predictions, "dist.hist": histogram, "dist.qqplot": QQ-plot) \& "dist.qqplot","dist.hist") &\\
\st{\ttfamily box} & If TRUE, boxplots are produced instead of scatterplots & FALSE \\
& & \\
\hline
\end{tabularx} 
\end{center}
%\caption{\itshape Default graphical parameters.\\ Any option not defined by the user is automatically set to its default value.}
%\label{tab::default_graph_param}
\end{table} 

%%%%%%%%%%%%%%%%%%%%%%%%%%%%%%%%%%%%%%%%%%%%%%%%%%%%%%%%%%%%%%%%%%%%%%%%%%%%%%%%%%%%%%
%%%%%%%%%%%%%%%%%%%%%%%%%%%%%%%%%%%%%%%%%%%%%%%%%%%%%%%%%%%%%%%%%%%%%%%%%%%%%%%%%%%%%%
 

\begin{table}[H] 
\vspace{-2.5cm}
\begin{center}
%\begin{tabular}{r p{10cm} p{3cm}}
  \begin{adjustbox}{max width=\textwidth}
\begin{tabularx}{\textwidth}[t]{|C p{7cm}  p{3cm}|C C |}
\arrayrulecolor{black}\hline
& \centering {\textbf{\textcolor{black}{Options for colours and line types}}} & \\
\centering{\textbf{\textcolor{black}{Parameter}} }& \centering{\textbf{\textcolor{black}{Description }}} & \textbf{\textcolor{black}{Default value}} \\
\hline
{ col} & Symbol and line colour for observed data & black \\
& applies to col.pobs and col.lobs if not given & \\
{\ttfamily lty} & Line type for observed data & 1 (straight line) \\
{\ttfamily lwd} & Line width for observed data & 0.8 \\
\textcolor{red}{\ttfamily  size } & Size for observations & 1.2 \\
\textcolor{red}{alpha} & Default transparency & 1\\

\st{\ttfamily col.lobs} & Symbol colour for observations (lines) & steelblue4 & \textcolor{blue}{\ttfamily remettre} \\
{\ttfamily col.pobs} & Symbol colour for observations (points) & steelblue4 \\
{\ttfamily pch.pobs} & Default symbol type & 20 (dot) \\
\st{\ttfamily lty.lobs} & Line type for observations & 1 & \textcolor{blue}{\ttfamily deprecated (lty) }\\
\st{\ttfamily lwd.lobs} & Line width for observations & 1 & \textcolor{blue}{\ttfamily deprecated (lwd)} \\
\st{\ttfamily  alpha.pobs} &  Transparency for observations & & \textcolor{blue}{\ttfamily deprecated (alpha)}\\
{\ttfamily col.pcens} & Symbol colour for censored observations & red \\
{\ttfamily pch.pcens} & Default symbol type for censored observations & 8 () \\
\textcolor{red}{\ttfamily  size.pcens  } & size for censored observations & 1.2 \\
\textcolor{red}{\ttfamily  alpha.pcens } & Transparency for censored observations  & 1 \\

\st{\ttfamily col.abline} & Colour of the horizontal/vertical lines added to the plots & "DarkBlue" & \textcolor{blue}{\ttfamily deprecated}\\
\st{\ttfamily lty.abline} & Type of the lines added to the plots & 2 (dashed)  & \textcolor{blue}{\ttfamily deprecated}\\
\st{\ttfamily lwd.abline} & Width of the lines added to the plots & 2 & \textcolor{blue}{\ttfamily deprecated} \\

\textcolor{red}{\ttfamily  col.x50} & Colour for the median line for loq plot &  & \textcolor{blue}{\ttfamily col.bands} \\
& Colour of the prediction bands in the loq plot & & \textcolor{blue}{\ttfamily fill.bands} \\
\textcolor{red}{\ttfamily  lty.x50} & Type of the median prediction line in loq plot & & \textcolor{blue}{\ttfamily lty.bands}  \\
\textcolor{red}{\ttfamily  lwd.x50} & Width of the median prediction line in loq plot & & \textcolor{blue}{\ttfamily lwd.bands} \\
\textcolor{red}{\ttfamily  alpha.x50} & Transparency of the median prediction line in loq plot & & \textcolor{blue}{\ttfamily alpha.bands} \\
\hline

\end{tabularx} 
\end{adjustbox}
\end{center}
%\caption{\itshape Default graphical parameters.\\ Any option not defined by the user is automatically set to its default value.}
%\label{tab::default_graph_param}
\end{table} 


\begin{table}[H] 
\vspace{-2.5cm}
\begin{center}
%\begin{tabular}{r p{10cm} p{3cm}}
  \begin{adjustbox}{max width=\textwidth}
\begin{tabularx}{\textwidth}[t]{|C p{7cm}  p{3cm}|C c|}
\arrayrulecolor{black}\hline
& \centering {\textbf{\textcolor{black}{Options for colours and line types}}} & \\
\centering{\textbf{\textcolor{black}{Parameter}} }& \centering{\textbf{\textcolor{black}{Description }}} & \textbf{\textcolor{black}{Default value}} \\
\hline
    
\textcolor{red}{\ttfamily     bar.bands.color} & Colour of the line around the prediction bars of an histogram & & \textcolor{blue}{\ttfamily col.bands}\\
\textcolor{red}{\ttfamily     bar.bands.fill} & Colour of the prediction bars of an histogram  & & \textcolor{blue}{\ttfamily fill.bands}\\
\textcolor{red}{\ttfamily     bar.bands.alpha} & Transparency of the prediction bars of an histogram & & \textcolor{blue}{\ttfamily alpha.bands}\\
\textcolor{red}{\ttfamily     bar.bands.lty} & Type of the line around the prediction bars of an histogram & & \textcolor{blue}{\ttfamily lty.bands}\\
\textcolor{red}{\ttfamily     bar.bands.lwd} &  Size of the line around the prediction bars of an histogram & & \textcolor{blue}{\ttfamily lwd.bands}\\

\textcolor{red}{\ttfamily     bar.color} & Colour of the line around the bars of an histogram & & \textcolor{blue}{\ttfamily col}\\
\textcolor{red}{\ttfamily     bar.fill} & Colour of the bars of an histogram & & \textcolor{blue}{\ttfamily fill}\\
\textcolor{red}{\ttfamily     bar.alpha} & Transparency of the bars of an histogram & & \textcolor{blue}{\ttfamily alpha} \\
\textcolor{red}{\ttfamily     bar.lty} & Type of the line around the bars of an histogram& & \textcolor{blue}{\ttfamily lty} \\
\textcolor{red}{ \ttfamily    bar.lwd} & Size of the line around the bars of an histogram& & \textcolor{blue}{\ttfamily lwd} \\
& & \\
{\ttfamily col.fillpi} & Colour used to fill histograms and prediction bands & slategray1 & \textcolor{blue}{\ttfamily fill.bands}\\

{\ttfamily col.fillmed} & Colour used to fill prediction band on the median (VPC, npde) & pink & \textcolor{blue}{\ttfamily fill.med}\\
{\ttfamily col.lmed} & Colour used to plot the predicted median (VPC, npde) & indianred4 & \textcolor{blue}{\ttfamily col.med}\\
{\ttfamily col.lpi} & Colour used to plot lower and upper quantiles & slategray4 & \textcolor{blue}{\ttfamily col.bands}\\
{\ttfamily lty.lmed} & Line type used to plot the predicted median (VPC, npde) & 2 & \textcolor{blue}{\ttfamily lty.med}\\
{\ttfamily lty.lpi} & Line type used to plot lower and upper quantiles & 2 & \textcolor{blue}{\ttfamily lty.bands}\\
{\ttfamily lwd.lmed} & Line width used to plot the predicted median (VPC, npde) & 1 & \textcolor{blue}{\ttfamily lwd.med}\\
{\ttfamily lwd.lpi} & Line width used to plot lower and upper quantiles & 1 & \textcolor{blue}{\ttfamily lwd.bands}\\
\hline
\end{tabularx} 
\end{adjustbox}

\end{center}
%\caption{\itshape Default graphical parameters.\\ Any option not defined by the user is automatically set to its default value.}
%\label{tab::default_graph_param}
\end{table} 


\begin{table}[H] 
\vspace{-2.5cm}
\begin{center}
%\begin{tabular}{r p{10cm} p{3cm}}
  \begin{adjustbox}{max width=\textwidth}
\begin{tabularx}{\textwidth}[t]{|C p{7cm}  p{3cm}|C c|}
\arrayrulecolor{black}\hline
& \centering {\textbf{\textcolor{black}{Options for colours and line types}}} & \\
\centering{\textbf{\textcolor{black}{Parameter}} }& \centering{\textbf{\textcolor{black}{Description }}} & \textbf{\textcolor{black}{Default value}} \\
\hline

\textcolor{red}{ \ttfamily    alpha.bnds} & Transparency of the confidence interval for the plot & & \textcolor{blue}{\ttfamily alpha.bands}\\
\textcolor{red}{ \ttfamily    fillcolor.bnds} & Colour of the confidence interval for the plot & & \textcolor{blue}{\ttfamily fill.bands}\\
\textcolor{red}{ \ttfamily    col.bnds} &  Colour  for the lines of the mean and bounds of the confidence interval for the plot  & & \textcolor{blue}{\ttfamily col.bands}\\
\textcolor{red}{ \ttfamily    lty.bnds} &  Type for the lines of the mean and bounds of the confidence interval for the plot & & \textcolor{blue}{\ttfamily lty.bands}\\
\textcolor{red}{ \ttfamily    lwd.bnds} & Width of the lines of the mean and bounds of the confidence interval for the plot & & \textcolor{blue}{\ttfamily lwd.bands}\\
\textcolor{red}{ \ttfamily    alpha.bnds.median}&  Transparency  for the color of the confidence interval for the plot  & & \textcolor{blue}{\ttfamily alpha.med}\\
\textcolor{red}{ \ttfamily    fillcolor.bnds.median} & Colour for the median prediction band & & \textcolor{blue}{\ttfamily fill.med}\\
\textcolor{red}{ \ttfamily    col.bnds.median} &  Colour for lines of the boundaries of the median prediction band & & \textcolor{blue}{\ttfamily col.med}\\
\textcolor{red}{ \ttfamily    lty.bnds.median} &  Type for the lines the boundaries of the median prediction band & & \textcolor{blue}{\ttfamily lty.med}\\
\textcolor{red}{ \ttfamily    lwd.bnds.median} & Width of the the lines the boundaries of the median prediction band   & & \textcolor{blue}{\ttfamily lwd.med}\\
\textcolor{red}{ \ttfamily    alpha.bnds.upp.low} & Transparency for the lines the bounds of the median prediction band  & & \textcolor{blue}{\ttfamily alpha.bands}\\
\textcolor{red}{ \ttfamily    fillcolor.bnds.upp.low} & Colour for the upper and lower prediction band  & & \textcolor{blue}{\ttfamily fill.bands}\\
\textcolor{red}{ \ttfamily    col.bnds.upp.low} & Colour for lines of the bounds of the upper and lower  prediction band & & \textcolor{blue}{\ttfamily col.bands}\\
\textcolor{red}{ \ttfamily    lty.bnds.upp.low} & Type for lines of the bounds of the upper and lower  prediction band & & \textcolor{blue}{\ttfamily lty.bands}\\
\textcolor{red}{ \ttfamily    lwd.bnds.upp.low} & Width of the lines of the bounds of the upper and lower  prediction band & & \textcolor{blue}{\ttfamily lwd.bands}\\
\textcolor{red}{ \ttfamily    col.line.pred.median} & Colour for lines of the mean prediction  & & \textcolor{blue}{\ttfamily col.med} \\
\textcolor{red}{ \ttfamily    lty.line.pred.median} &  Type for lines of the mean prediction& & \textcolor{blue}{\ttfamily lty.med} \\
\textcolor{red}{ \ttfamily    lwd.line.pred.median} &  Size  for lines of the mean prediction& & \textcolor{blue}{\ttfamily lwd.med} \\
\textcolor{red}{ \ttfamily    col.line.pred.upp.low} &  Colour for lines of the upper and lower prediction& & \textcolor{blue}{\ttfamily col.bands} \\
\textcolor{red}{ \ttfamily    lty.line.pred.upp.low} &  Type for lines of the upper and lower prediction& & \textcolor{blue}{\ttfamily lty.bands} \\
\textcolor{red}{ \ttfamily    lwd.line.pred.upp.low} &  Size for lines of the upper and lower prediction& & \textcolor{blue}{\ttfamily lwd.bands} \\
\hline
\end{tabularx} 
\end{adjustbox}

\end{center}
%\caption{\itshape Default graphical parameters.\\ Any option not defined by the user is automatically set to its default value.}
%\label{tab::default_graph_param}
\end{table} 


%%%%%%%%%%%%%%%%%%%%%%%%%%%%%%%%%%%%%%%%%%%%%%%%%%%%%%%%%%%%%%%%%%%%%%%%%%%%%%%%%%%%%%
%%%%%%%%%%%%%%%%%%%%%%%%%%%%%%%%%%%%%%%%%%%%%%%%%%%%%%%%%%%%%%%%%%%%%%%%%%%%%%%%%%%%%%
 

\begin{table}[H] 
\begin{center}
%\begin{tabular}{r p{10cm} p{3cm}}
\begin{tabularx}{\textwidth}[t]{|C p{7cm} C| c}
\arrayrulecolor{black}\hline
& \centering {\textbf{\textcolor{black}{Graphical options for VPC and residual plots}}} & \\
\centering{\textbf{\textcolor{black}{Parameter}} }& \centering{\textbf{\textcolor{black}{Description }}} & \textbf{\textcolor{black}{Default value}} \\
\hline
{\ttfamily bands} & Whether prediction intervals should be plotted & TRUE \\
{\ttfamily approx.pi} & If TRUE, samples from $\mathcal{N}(0,1)$ are used to plot prediction intervals, while if FALSE, prediction bands are obtained using pd/npde computed for the simulated data & TRUE \\
{\sf vpc.method} & Method used to bin points (one of "equal", "width", "user" or "optimal"); at least the first two letters of the method need to be specified & "equal" \\
{\ttfamily vpc.bin} & Number of binning intervals & 10 \\
{\ttfamily vpc.interval} & Size of interval & 0.95 \\
{\ttfamily vpc.breaks} & Vector of breaks used with user-defined breaks (vpc.method="user") & NULL \\
{\ttfamily vpc.extreme} & Can be set to a vector of 2 values to fine-tune the behaviour of the binning algorithm at the boundaries; specifying c(0.01,0.99) with the "equal" binning method and vpc.bin=10 will create 2 extreme bands containing 1\% of the data on the X-interval, then divide the region within the two bands into the remaining 8 intervals each containing the same number of data; in this case the intervals will all be equal except for the two extreme intervals, the size of which is fixed by the user; complete fine-tuning can be obtained by setting the breaks with the vpc.method="user" & NULL \\
{\ttfamily pi.size} & Width of the prediction interval on the quantiles & 0.95 \\
{\ttfamily vpc.lambda} & Value of lambda used to select the optimal number of bins through a penalised criterion & 0.3 \\
{\ttfamily vpc.beta} & Value of beta used to compute the variance-based criterion (Jopt,beta(I)) in the clustering algorithm & 0.2 \\
{\ttfamily bands.rep} & Number of simulated datasets used to compute prediction bands & 200 \\
\hline
\end{tabularx} 
\end{center}
%\caption{\itshape Default graphical parameters.\\ Any option not defined by the user is automatically set to its default value.}
%\label{tab::default_graph_param}
\end{table} 

%
%\clearpage
%
%\begin{lstlisting}[ basicstyle=\small\color{black}\ttfamily]
%
%plot(xtheo_cens,plot.type="x.scatter",which="npde",
%      
%      main.title = "npde vs time data xtheo_cens",
%      size.main.title = 14,
%      sub.title = "", # no sub.title 
%      size.sub.title = "", 
%      
%      xlab= "Time",
%      ylab= "npde",
%      size.xlab = 12, 
%      size.ylab = 12,
%      xlim=c(), # by default
%      ylim=c(), # by default
%      approx.pi=TRUE,
%      bands=TRUE,
%      plot.obs=TRUE,
%      
%      alpha.bnds.median = 0.25,
%      fillcolor.bnds.median = "firebrick4",
%      col.bnds.median="red",
%      lty.bnds.median=3,
%      lwd.bnds.median=1,     
%      alpha.bnds.upp.low = 0.25,
%      fillcolor.bnds.upp.low = "dodgerblue",
%      col.bnds.upp.low="green",
%      lty.bnds.upp.low=6,
%      lwd.bnds.upp.low=1,      
%      col.line.pred.median = "red",
%      lty.line.pred.median = 1,
%      lwd.line.pred.median = 1,      
%      col.line.pred.upp.low = "blue",
%      lty.line.pred.upp.low = 1,
%      lwd.line.pred.upp.low = 1,
%      
%      col.pobs = "orangered3",
%      pch.pobs = 12,
%      size.pobs = 1.5,      
%      col.pencs = "yellow",
%      pch.pencs = 15,
%      size.pcens = 1.75,
%      
%      size.text.x = 10,
%      size.text.y = 10,    
%      x.breaks = 10,
%      y.breaks = 10, 
%      xlog = FALSE,
%      ylog = FALSE)
% \end{lstlisting}
%
%
%\begin{figure}[t]
%\centering     
%\subfigure[]{\label{fig:xtheo_cens_xscatter}\includegraphics[width=150mm]{xtheo_cens_xscatter.pdf}}
%\caption{xtheo\_cens\_xscatter.}
%\label{fig:xtheo_cens_xscatter}
%\end{figure}


\end{document}
