\section{Graph options}

\subsection{Types of graphs}

\hskip 18pt Table~II shows which plot types are available (some depend on whether for instance covariates or data below the limit of quantification are present in the dataset) for a {\sf NpdeObject} object. 
\begin{table}[!h]
\noindent{\bfseries Table II:} {\itshape Types of plots available.}
%\label{tab:plot.types}
\begin{center}
\begin{tabular} {r p{10cm}}
\hline {\bf Plot type} & {\bf Description} \\
\hline
data & Plots the observed data in the dataset \\
x.scatter & Scatterplot of the \npd~versus the predictor X (optionally can plot \pd~or \npde~instead) \\
pred.scatter & Scatterplot of the \npd~versus the population predicted values \\
cov.scatter & Scatterplot of the \npd~versus covariates \\
covariates & \npd~represented as boxplots for each covariate category \\
vpc & Plots a Visual Predictive Check \\
loq & Plots the probability for an observation to be BQL, versus the predictor X \\
ecdf & Empirical distribution function of the \npd (optionally \pd~or \npde) \\
hist & Histogram of the \npd (optionally \pd~or \npd) \\
qqplot & QQ-plot of the \npd versus its theoretical distribution (optionally \pd~or \npde) \\
cov.x.scatter & Scatterplot of the \npd~versus the predictor X (optionally can plot \pd~or \npde~instead), split by covariate category \\
cov.pred.scatter & Scatterplot of the \npd~versus the population predicted values, split by covariate category \\
cov.hist & Histogram of the \npd, split by covariate category \\
cov.ecdf & Empirical distribution function of the \npd, split by covariate category \\
cov.qqplot & QQ-plot of the \npd, split by covariate category \\
\hline
\end{tabular}
\end{center}
%\caption{Plot types available.} 
\end{table}
These different plots are available using the option {\sf plot.type}, as in:
\begin{verbatim}
plot(x,plot.type="data")
\end{verbatim}

The plots are all produced by default for \npd, but can be produced for \npde~or \pd~using the {\sf which} argument. The final five plots can also be accessed with the base plot and the option {\sf covsplit=TRUE}. For instance, \verb+ plot(x,plot.type="cov.x.scatter")+ is equivalent to \verb+ plot(x,plot.type="x.scatter",covsplit=TRUE)+.

Given an object {\sf x} resulting from a call to {\sf npde} or {\sf autonpde}, default plots can be produced using the following command (see figure~\ref{fig:respde} for instance):
\begin{verbatim}
plot(x)
\end{verbatim}
This graph can also be produced using the individual plots and arranging them in a 2x2 layout:
\begin{verbatim}
x1<-plot(x, plot.type="hist")
x2<-plot(x, plot.type="qqplot")
x3<-plot(x, plot.type="x.scatter")
x4<-plot(x, plot.type="pred.scatter")
grid.arrange(grobs=list(x1,x2,x3,x4), nrow=2, ncol=2)
\end{verbatim}

\subsection{Options for graphs}

\hskip 18pt The default layout for graphs in the {\sf npde} library can be modified through the use of many options. An additional document, \verb+demo_npde3.0.pdf+, is included in the \texttt{inst} directory of the package, presenting additional examples of graphs and how to change the options.

Table~III following table shows the options that can be set, either by specifying them on the fly in a call to plot applied to a NpdeObject object, or by storing them in the {\sf prefs} component of the object.

Note that not all of the graphical parameters in \texttt{par()} can be used, but it is possible for instance to use the {\sf xaxt="n"} option below to suppress plotting of the X-axis, and to then add back the axis with the \R~function {\sf axis()} to tailor the tickmarks or change colours as wanted. It is also possible of course to extract \npde, fitted values or original data to produce any of these plots by hand if the flexibility provided in the library isn't sufficient. Please refer to the document \verb+demo_npde3.0.pdf+ for examples of graphs using these options.

The arguments can be set when calling the plot, for instance:
\begin{verbatim}
plot(x, main="Default npde plots")
plot(x, plot.type="x.scatter", bin.method="optimal", main="Optimal binning method for scatterplot")
\end{verbatim}
Some of these parameters however will only work in some graphs and will be ignored otherwise. For example, the argument {\sf fill.med} (table IV) controls the colour of the band for the median percentile of observed data, and this colour is not used in histograms and qq-plots where fill.bands is used to colour the prediction bands around the distribution.

\newpage

\begin{table}[!h] 
\begin{center}
\begin{tabular}{| r p{8cm} c|}
\hline
\textbf{\textcolor{black}{Argument}} & \centering{\textbf{\textcolor{black}{Description }}} & \textbf{\textcolor{black}{Default value}} \\
\hline
{\ttfamily verbose} & Output is produced for some plots (most notably when binning is used, this prints out the boundaries of the binning intervals) if TRUE & FALSE \\
{\ttfamily main} & Title & depends on plot \\
{\ttfamily sub } & Subtitle & empty \\
{\ttfamily size.main } & Size of the main title & 14 \\
{\ttfamily size.sub  } & Size of the title for covariate & 12 \\

{\ttfamily xlab} & Label for the X-axis & depends on plot \\
{\ttfamily ylab} & Label for the Y-axis & depends on plot \\
{\ttfamily size.xlab} & Size of the label for the X-axis & 12 \\
{\ttfamily size.ylab} & Size of the label for the Y-axis & 12 \\
{\ttfamily breaks.x} & Number of tick marks on the X-axis & 10 \\
{\ttfamily breaks.y} & Number of tick marks on the Y-axis & 10 \\
{\ttfamily size.x.text} & Size of tick marks and tick labels on the X-axis & 10 \\
{\ttfamily size.y.text} & Size of tick marks and tick labels on the Y-axis & 10 \\

{\ttfamily xlim} & Range of values on the X-axis & empty, adjusts to the data \\
{\ttfamily ylim} & Range of values on the Y-axis & empty, adjusts to the data \\

{\ttfamily xaxt} & A character whether to plot the X axis. Specifying "n" suppresses plotting of the axis & "y"  \\
{\ttfamily yaxt} & A character whether to plot the Y axis. Specifying "n" suppresses plotting of the axis & "y" \\

{\ttfamily xlog} & Scale for the X-axis (TRUE: logarithmic scale) & FALSE \\
{\ttfamily ylog} & Scale for the Y-axis (TRUE: logarithmic scale) & FALSE \\
 {\ttfamily grid } & If TRUE, display a grid on the background of the plot & FALSE \\
{\ttfamily } & &  \\
\hline
\end{tabular} 
\end{center}
\noindent{\bfseries Table III:} {\itshape Layout, titles and axes.}
\end{table} 


\clearpage

\begin{table}[!h] 
\begin{center}
\begin{tabular}{| r p{8cm} c|}
\hline
\textbf{\textcolor{black}{Argument}} & \centering{\textbf{\textcolor{black}{Description }}} & \textbf{\textcolor{black}{Default value}} \\
\hline
 {\ttfamily plot.obs } & If TRUE, observations, pd/ndpe should are plotted on top of the prediction bands & TRUE \\
 {\ttfamily plot.box } & If TRUE, boxplots are produced instead of scatterplots & FALSE \\
 {\ttfamily covsplit } & If TRUE, plot are split by covariates & FALSE \\
 {\ttfamily plot.loq } & If TRUE, data under the LOQ are plotted & TRUE \\
 {\ttfamily line.loq } & If TRUE, horizontal line should is plotted at Y=LOQ in data and VPC plots & FALSE \\
 {\ttfamily impute.loq } & If TRUE, the imputed values are plotted for data under the LOQ & TRUE \\
{\ttfamily } & &  \\
\hline
\end{tabular} 
\end{center}
\noindent{\bfseries Table IV:} {\itshape Parameters controlling content.} %\label{tab:graphicalOptions5}
\end{table} 

\begin{table}[!h] 
\begin{center}
\begin{tabular}{|r p{10cm} p{3cm} |}
\hline
\centering{\textbf{\textcolor{black}{Parameter}} }& \centering{\textbf{\textcolor{black}{Description }}} & \textbf{\textcolor{black}{Default value}} \\
\hline
{\ttfamily bands} & Whether prediction intervals should be plotted & TRUE \\
{\ttfamily approx.pi} & If TRUE, samples from $\mathcal{N}(0,1)$ are used to plot prediction intervals, while if FALSE, prediction bands are obtained using pd/npde computed for the simulated data & TRUE \\
{\ttfamily bin.method} & Method used to bin points (one of "equal", "width", "user" or "optimal"); at least the first two letters of the method need to be specified & "equal" \\
{\ttfamily bin.number} & Number of binning intervals & 10 \\
{\ttfamily vpc.interval} & Size of interval & 0.95 \\
{\ttfamily bin.breaks} & Vector of breaks used with user-defined breaks (vpc.method="user") & NULL \\
{\ttfamily bin.extreme} & Can be set to a vector of 2 values to fine-tune the behaviour of the binning algorithm at the boundaries; specifying c(0.01,0.99) with the "equal" binning method and vpc.bin=10 will create 2 extreme bands containing 1\% of the data on the X-interval, then divide the region within the two bands into the remaining 8 intervals each containing the same number of data; in this case the intervals will all be equal except for the two extreme intervals, the size of which is fixed by the user; complete fine-tuning can be obtained by setting the breaks with the vpc.method="user" & NULL \\
{\ttfamily pi.size} & Width of the prediction interval on the quantiles & 0.95 \\
{\ttfamily bin.lambda} & Value of lambda used to select the optimal number of bins through a penalised criterion & 0.3 \\
{\ttfamily bin.beta} & Value of beta used to compute the variance-based criterion (Jopt,beta(I)) in the clustering algorithm & 0.2 \\
{\ttfamily bands.rep} & Number of simulated datasets used to compute prediction bands & 200 \\
\hline
\end{tabular} 
\end{center}
\noindent{\bfseries Table V:} {\itshape Graphical options for VPC and residual plots.} %\label{tab:graphicalOptions4}
\end{table} 

\begin{table}[!h] 
\vspace{-2cm}
\begin{center}
\begin{tabular}{| r p{8cm} c|}
\hline
\textbf{\textcolor{black}{Argument}} & \centering{\textbf{\textcolor{black}{Description }}} & \textbf{\textcolor{black}{Default value}} \\
\hline
{\ttfamily col} & Main colour for observed data (applied to lines and symbols pertaining to observations if no other option is given to supersede this value) & "slategray4"  \\
{\ttfamily lty} & Line type for observed data & 1 \\
{\ttfamily lwd} & Line width for observed data & 0.5 \\
{\ttfamily pch} & Symbol used to plot observed data &  20 \\
{\ttfamily alpha} & Transparencyfor observed data  & 1 \\
{\ttfamily size} & Symbol size to plot observed data & 1  \\
{\ttfamily fill} & Colour used to fill area elements related to observed data (such as histogram bars) & "white  \\
{\ttfamily } & &  \\
{\ttfamily type } &  Type for the line for qqplot and scatter. Display line and points. & "b"  \\
{\ttfamily col.pobs} & Colour for observed data & "slategray4"  \\
{\ttfamily pch.pobs} & Symbol used to plot observed data &  20 \\
{\ttfamily size.pobs} & Symbol size to plot observed data & 1.5  \\
{\ttfamily alpha.pobs} & Transparency for observed data  & 0.5  \\
{\ttfamily } & &  \\

{\ttfamily col.lobs} & Colour for the line of observed data & "slategray4"  \\
{\ttfamily lty.lobs} & Line type for the line of observed data &  1 \\
{\ttfamily lwd.lobs} & Line width  for the line of observed data & 0.5  \\
{\ttfamily } & &  \\
{\ttfamily col.pcens} & Colour for the censored data  & "steelblue3"  \\
{\ttfamily pch.pcens} & Symbol for the censored data  &  8 \\
{\ttfamily size.pcens} & Symbol size for the censored data  &  0.6 \\
{\ttfamily alpha.pcens} &Transparency for the censored data & 1  \\
{\ttfamily } & &  \\
{\ttfamily col.line.loq} & Colour for the LOQ line  & "black"  \\
{\ttfamily lty.line.loq} & Symbol type for the LOQ line &  5 \\
{\ttfamily lwd.line.loq} & Symbol size for the LOQ line &  0.5 \\
{\ttfamily } & &  \\
\hline
\end{tabular} 
\end{center}
\noindent{\bfseries Table V:} {\itshape Colours, transparency, line types and symbols.}
\end{table} 

\clearpage

\begin{table}[!h] 
\begin{center}
\begin{tabular}{| r p{8cm} c|}
\hline
\textbf{\textcolor{black}{Argument}} & \centering{\textbf{\textcolor{black}{Description }}} & \textbf{\textcolor{black}{Default value}} \\
\hline
{\ttfamily fill.outliers.med} & Color for the outliers of the median confidence interval & "red"  \\
{\ttfamily fill.outliers.bands} & Color for the outliers of  the bounds of the confidence interval & "red"  \\
{\ttfamily alpha.outliers.med} & Transparency of the color for the outliers of the median confidence interval & 1  \\
{\ttfamily alpha.outliers.bands} & Transparency of the color  for the outliers the bounds of the confidence interval & 1  \\
{\ttfamily col.bands} &  Colour for the lines of the bounds of the confidence interval & "white"  \\
{\ttfamily lty.bands} &  Type for the lines of bounds of the confidence interval & 2  \\
{\ttfamily lwd.bands} &  Width of the lines of bounds of the confidence interval & 0.25  \\
{\ttfamily alpha.bands} &  Transparency of the bounds of the confidence interval & 0.3  \\
{\ttfamily fill.bands} &  Colour of the confidence interval & "steelblue2"  \\
{\ttfamily } & &  \\
{\ttfamily col.med} &  Colour for the lines of the median of the confidence interval & "white"  \\
{\ttfamily lty.med} &  Type for the lines of the median
of the confidence interval & 2  \\
{\ttfamily lwd.med} &  Width of the lines of the median
of the confidence interva & 0.5  \\
{\ttfamily alpha.med} &  Transparency of the median confidence interval & 0.5  \\
{\ttfamily fill.med} &  Colour of the median confidence  interval & "pink"  \\
{\ttfamily } & &  \\
{\ttfamily col.ther} &  Colour for the lines for model-derived percentiles &  \\
{\ttfamily lty.ther} &  Type for the lines for model-derived percentilesl & 2  \\
{\ttfamily lwd.ther} &  Width of the lines for model-derived percentiles & 0.5  \\
{\ttfamily alpha.ther} &  Transparency of the lines for model-derived percentiles & 0.6  \\
\hline
\end{tabular} 
\end{center}
\noindent{\bfseries Table V (cont):} {\itshape Colours, transparency, line types and symbols.} %\label{tab:graphicalOptions3}
\end{table} 
